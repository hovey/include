% version 8 Oct 1998 
% Stanford University
% formerly Lausanne, Switzerland


% *-----------------------*
% | LaTeXING THE DOCUMENT |
% *-----------------------*--------------------------------------
%
% Run LaTeX on this file twice for proper section numbers.
% A '%' causes LaTeX to ignore remaining text on the line
% latex filename.tex
% dvips filename.dvi -o filename.ps
% ghostview filename.ps 
%
% *--------------------------------------------------------------

% Statistics
\newcommand{\E}{\mathbb{E}}
\newcommand{\ex}{\mathbb{E}[X]}
\newcommand{\ey}{\mathbb{E}[Y]}
\newcommand{\exy}{\mathbb{E}[XY]}
\newcommand{\cov}{\mbox{\sc Cov}}
\newcommand{\var}{\mbox{\sc Var}}

\newcommand{\eg}{{\em e.g., }}
\newcommand{\ie}{{\em i.e., }}
\newcommand{\viz}{{\em viz.}}
\newcommand{\htable}{{\sc Table}}
\newcommand{\hfigure}{{\sc Figure}}

\newcommand{\hframe}[1]{{}^{\mathcal{#1}}\!}

% PLASTICITY - formulation
% ============
 \newcommand{\hstrain}{\varepsilon}
 \newcommand{\vhstrain}{\ve{\varepsilon}}
 \newcommand{\hstraindot}{\dot{\varepsilon}}
 \newcommand{\hstraine}{\varepsilon^{\mbox{\footnotesize e}}}
 \newcommand{\hstrainp}{\varepsilon^{\mbox{\footnotesize p}}}
 \newcommand{\hstrainpdot}{\dot{\varepsilon}^{\mbox{\footnotesize p}}}
 \newcommand{\gammadot}{\dot{\gamma}}
 \newcommand{\sign}{\mbox{sign}}
 \newcommand{\sigmadot}{\dot{\sigma}}  
 \newcommand{\vsigma}{\ve{\sigma}}  
 \newcommand{\vsigmadot}{\dot{\ve{\sigma}}}  
 \newcommand{\vsigmavol}{\ve{\sigma}_{\tiny\mbox{VOL}}}
 \newcommand{\vsigmadev}{\ve{\sigma}_{\tiny\mbox{DEV}}}
 \newcommand{\Phidot}{\dot{\Phi}}
 \newcommand{\setequal}{\stackrel{\mbox{set}}{=}}
 \newcommand{\Cep}{C_{\mbox{\footnotesize ep}}}
 \newcommand{\Eep}{E_{\mbox{\footnotesize ep}}}
 \newcommand{\keq}{k_{\mbox{\footnotesize eq}}}
 \newcommand{\alphadot}{\dot{\alpha}}

% PLASTICITY - implementation
% ============
 \newcommand{\hstrainn}{\varepsilon_n}
 \newcommand{\hstrainen}{\varepsilon^{\mbox{\footnotesize e}}_{n}}
 \newcommand{\hstrainpn}{\varepsilon^{\mbox{\footnotesize p}}_{n}}
 \newcommand{\alphan}{\alpha_n}
 \newcommand{\gamman}{\gamma_n}
 \newcommand{\hstrainnp}{\varepsilon_{n+1}}
 \newcommand{\hstrainenp}{\varepsilon^{\mbox{\footnotesize e}}_{n+1}}
 \newcommand{\hstrainpnp}{\varepsilon^{\mbox{\footnotesize p}}_{n+1}}
 \newcommand{\hstrainpnpi}{\varepsilon^{\mbox{\footnotesize p}\;(i)}_{n+1}}
 \newcommand{\hstrainpnpip}{\varepsilon^{\mbox{\footnotesize p}\;(i+1)}_{n+1}}
 \newcommand{\hstrainptrnp}{\varepsilon^{\mbox{\footnotesize p tr}}_{n+1}}
 \newcommand{\alphanp}{\alpha_{n+1}}
 \newcommand{\alphanpi}{\alpha_{n+1}^{(i)}}
 \newcommand{\alphatrnp}{\alpha^{\mbox{\footnotesize tr}}_{n+1}}
 \newcommand{\alphanpip}{\alpha_{n+1}^{(i+1)}}
 \newcommand{\gammanp}{\gamma_{n+1}}
 \newcommand{\gammadotnp}{\dot{\gamma}_{n+1}} 

 \newcommand{\sigmatrnp}{\sigma^{\mbox{\footnotesize tr}}_{n+1}}
 \newcommand{\sigman}{\sigma_n}
 \newcommand{\sigmanp}{\sigma_{n+1}}

 \newcommand{\tn}{t_n}
 \newcommand{\tnp}{t_{n+1}}

 \newcommand{\Phinp}{\Phi_{n+1}}
 \newcommand{\Phidotnp}{\dot{\Phi}_{n+1}}
 \newcommand{\Phitrnp}{\Phi^{\mbox{\footnotesize tr}}_{n+1}}

% CIRCLED STATES
% ==============
\newcommand*\circled[1]{\tikz[baseline=(char.base)]{
            \node[shape=circle,draw,inner sep=2pt] (char) {#1};}}

% *--------------*
% | MATH SYMBOLS |
% *--------------*-----------------------------------------------
%
  \newcommand{\hcos}{\sf c}
  \newcommand{\hsin}{\sf s}
  \newcommand{\epi}{\mbox{epi }}
  \newcommand{\fa}{\forall \quad}
  \newcommand{\sotwo}{SO(2)}
  \newcommand{\sothree}{SO(3)}
% \newcommand {\angvel}[3]{{}^{#1}{#2}^{#3}}
% \newcommand{\implies}{\Rightarrow}
  \newcommand{\dirderiv}[1]{\mathbf{D}_{#1}}
  \newcommand{\abs}[1]{| \, #1 \, |}
  \newcommand{\innerproductmap}{\langle \, \cdot \, , \, \cdot \, \rangle}
  \newcommand{\innerproduct}[2]{\langle \, #1    \, , \, #2    \, \rangle}
  \newcommand{\normmap}{\parallel \, \cdot \, \parallel}
  \newcommand{\norm}[1]{\parallel~#1~\parallel}  
  \newcommand{\trace}{\mbox{tr}}
  \newcommand{\DEV}{\mbox{DEV}}
  \newcommand{\dev}{\mbox{dev}}
  \newcommand{\VOL}{\mbox{VOL}}
  \newcommand{\vol}{\mbox{vol}}
  \newcommand{\smDEV}{\mbox{\tiny DEV}}
  \newcommand{\smdev}{\mbox{\tiny dev}}
  \newcommand{\smVOL}{\mbox{\tiny VOL}}
  \newcommand{\smvol}{\mbox{\tiny vol}}
  \newcommand{\DEVoper}{\mbox{DEV}[\;\cdot\;]}
  \newcommand{\devoper}{\mbox{dev}[\;\cdot\;]}
  \newcommand{\ltwonorm}[1]{\parallel #1 \parallel_{_2}}
  \newcommand{\st}{\mbox{s.t. }}
  \newcommand{\pd}{\mbox{p.d. }}
  \newcommand{\defe}{\stackrel{\Delta}{=}}
  \newcommand{\dst}{\displaystyle} 
  \newcommand{\be}{\begin{equation}}
  \newcommand{\ee}{\end{equation}}
% \newcommand{\ve}[1]{\mathbf{#1}}
  \newcommand{\ve}[1]{\mbox{\boldmath $#1$}}
  \newcommand{\half}{\frac{1}{2}}
  \newcommand{\degrees}{^{\circ}}

% the old diad that I commented out 5/24/96
% \newcommand{\di}[1]{\mathsf{#1}}
% this new one seems to work, helvetica under the file
% /usr/local/teTeX/texmf/fonts/tfm/adobe/helvetic/phvb8r.tfm
%  \font\gm=msym10  % <- this one doesn't work
  \font\gm=phvb8r   % this is helvetica font essentially
  \newcommand{\di}[1]{\hbox{\gm #1}}
  \font\idm=zptmcmrm
  \newcommand{\dii}{\hbox{\idm I}}
  \font\zaph=pzcmi8t
  \newcommand{\refframe}[1]{\hbox{\zaph #1}}

% \newcommand{\di}[1]{\hbox{\tenbfss #1}}
  \newcommand{\dib}[1]{\bf \mathsf{#1} }
%  \newcommand{\qed}{{\;\; \blacksquare} }
%
% *--------------------------------------------------------------

% THERMODYNAMICS
% ============
\newcommand{\svol}{\mathcal{V}}  % specific volume
\newcommand{\senergy}{\mathcal{E}} % specific total energy
\newcommand{\IE}{\mathcal{I}} % internal energy
\newcommand{\KE}{\mathcal{K}} % kinetic energy
\newcommand{\PE}{\mathcal{P}} % potential energy
\newcommand{\helmholtz}{\mathcal{F}} % Helmholtz free energy
\newcommand{\gibbs}{\mathcal{G}} % Gibbs free energy
\newcommand{\enthalpy}{\mathcal{H}} % enthalpy


% *--------------------------------------------------------------

% BIOMECHANICS
% ============
\newcommand{\Pmtar}{P_{\mbox{\tiny mtar}}}
\newcommand{\Pheel}{P_{\mbox{\tiny heel}}}


% VECTOR SPACE defined by Gurtin
% ==============================
\newcommand{\lin}{{\sc lin}}
\newcommand{\linplus}{{\sc lin}$^+$}
\newcommand{\sym}{{\sc sym}}
% skew is already defined, define as skw instead
% \newcommand{\skew}{{\sc skw}}
\newcommand{\skw}{{\sc skw}}
\newcommand{\psym}{{\sc psym}}
\newcommand{\orth}{{\sc orth}}
\newcommand{\orthplus}{{\sc orth}$^+$}
\newcommand{\eig}{{\sc eig}}
%
 
% VECTOR CALCULUS OPERATORS
% =========================

\newcommand{\Grad}{\ve{\nabla}_{\hspace{-1mm}\mbox{\tiny 0}}}
\newcommand{\grad}{\ve{\nabla}}
\newcommand{\gradX}{\ve{\nabla}}
\newcommand{\gradx}{\ve{\nabla}_{x}}
\renewcommand{\div}{\cdot}
\newcommand{\vdiv}{\ve{\cdot}}
\newcommand{\ddiv}{:}
\newcommand{\vddiv}{\ve{:}}
\newcommand{\union}{\cup}
\newcommand{\intersection}{\cap}

\newcommand{\DIV}{\mbox{DIV}}
%\newcommand{\div}{\mbox{div}}


% DINGBATS (CIRLED NUMBERS)
% (MUST ALSO HAVE \usepackage{pifont} IN HEADER
% =============================================
\newcommand{\circleone}{{\mbox{\small \ding{172}}}}
\newcommand{\circletwo}{{\mbox{\small \ding{173}}}}

% VECTOR ALPHABET
% ===============

\newcommand{\vA}{\ve{A}}
\newcommand{\va}{\ve{a}}
\newcommand{\Dva}{\Delta \va}
\newcommand{\vB}{\ve{B}}
\newcommand{\vb}{\ve{b}}
\newcommand{\vC}{\ve{C}}
\newcommand{\vCdot}{\dot{\ve{C}}}
\newcommand{\vCinv}{\ve{C}^{-1}}
\newcommand{\vc}{\ve{c}}
\newcommand{\vD}{\ve{D}}
\newcommand{\vd}{\ve{d}}
\newcommand{\Dvd}{\Delta \vd}
\newcommand{\vE}{\ve{E}}
\newcommand{\vEdot}{\dot{\ve{E}}}
\newcommand{\vF}{\ve{F}}
\newcommand{\vFdot}{\dot{\ve{F}}}
\newcommand{\vf}{\ve{f}}
\newcommand{\vG}{\ve{G}}
\newcommand{\vg}{\ve{g}}
\newcommand{\gdot}{\dot{g}}
\newcommand{\vH}{\ve{H}}
\newcommand{\vh}{\ve{h}}
\newcommand{\vI}{\ve{I}}
\newcommand{\vid}{\ve{1}}
\newcommand{\vJ}{\ve{J}}
\newcommand{\vK}{\ve{K}}
%\newcommand{\vKe}{\ve{K}^{\mbox{\tiny e}}} % already defined
\newcommand{\vL}{\ve{L}}
\newcommand{\vl}{\ve{l}}
\newcommand{\vM}{\ve{M}}
\newcommand{\vN}{\ve{N}}
\newcommand{\vn}{\ve{n}}
\newcommand{\vnhat}{\hat{\ve{n}}}
\newcommand{\vO}{\ve{O}}
\newcommand{\vP}{\ve{P}}
\newcommand{\vp}{\ve{p}}
\newcommand{\vPelas}{\ve{P}^{\mbox{\tiny elas}}}
\newcommand{\vPvisc}{\ve{P}^{\mbox{\tiny visc}}}
\newcommand{\vQ}{\ve{Q}}
\newcommand{\vq}{\ve{q}}
\newcommand{\vqdot}{\dot{\ve{q}}}
\newcommand{\vqddot}{\ddot{\ve{q}}}
\newcommand{\vR}{\ve{R}}
\newcommand{\vr}{\ve{r}}
\newcommand{\vrhat}{\hat{\ve{r}}}
\newcommand{\vS}{\ve{S}}
\newcommand{\vSelas}{\ve{S}^{\mbox{\tiny elas}}} 
\newcommand{\vSvisc}{\ve{S}^{\mbox{\tiny visc}}}
\newcommand{\vs}{\ve{s}}
\newcommand{\vsdot}{\dot{\ve{s}}}
\newcommand{\vsddot}{\ddot{\ve{s}}}
\newcommand{\vshat}{\hat{\ve{s}}}
\newcommand{\vT}{\ve{T}}
\newcommand{\vTbar}{\bar{\ve{T}}}
\newcommand{\vThat}{\hat{\ve{T}}}
\newcommand{\vTT}{\vT_{\mbox{\tiny T}}}
\newcommand{\vt}{\ve{t}}
\newcommand{\vthat}{\hat{\ve{t}}}
\newcommand{\vU}{\ve{U}}
\newcommand{\vUbar}{\bar{\ve{U}}}
\newcommand{\vUdot}{\dot{\ve{U}}}
\newcommand{\vUddot}{\ddot{\ve{U}}}
\newcommand{\vu}{\ve{u}}
\newcommand{\DvU}{\Delta \vU}
\newcommand{\vV}{\ve{V}}
\newcommand{\vv}{\ve{v}}
\newcommand{\Dvv}{\Delta \vv}
\newcommand{\vW}{\ve{W}}
\newcommand{\vw}{\ve{w}}
\newcommand{\vX}{\ve{X}}
\newcommand{\vx}{\ve{x}}
\newcommand{\vxhat}{\hat{\ve{x}}}
\newcommand{\vY}{\ve{Y}}
\newcommand{\vy}{\ve{y}}
\newcommand{\vyhat}{\hat{\ve{y}}}
\newcommand{\vZ}{\ve{Z}}
\newcommand{\vz}{\ve{z}}

% VECTOR NUMBERS
% ==============
  \newcommand{\vzero}{\ve{0}}

% BLACKBOARD ALPHABET
% ===================
  \newcommand{\dist}{\mathbb{D}}
  \newcommand{\proj}{\mathbb{P}}
  \newcommand{\projT}{\proj_{\mbox{\tiny T}}}
  \newcommand{\interval}{\mathbb{T}}

% VECTOR GREEK ALPHABET
% =====================
  \newcommand{\vDelta}{\ve{\Delta}}
  \newcommand{\vdelta}{\ve{\delta}}
  \newcommand{\vkappa}{\ve{\kappa}}
  \newcommand{\vLambda}{\ve{\Lambda}}
  \newcommand{\vlambda}{\ve{\lambda}}
  \newcommand{\vphi}{\ve{\phi}}
  \newcommand{\vvarphi}{\ve{\varphi}}
  \newcommand{\vvarphidot}{\dot{\ve{\varphi}}}
  \newcommand{\vvarphiddot}{\ddot{\ve{\varphi}}}
  \newcommand{\vxi}{\ve{\xi}}
  \newcommand{\vthetahat}{\hat{\ve{\theta}}}
  \newcommand{\vtau}{\ve{\tau}}
  \newcommand{\pushforward}[1]{\varphi_*\left[\;#1\;\right]}
  \newcommand{\pullback}[1]{\varphi^{-1}_*\left[\;#1\;\right]}
  \newcommand{\lie}[1]{\mathcal L[\;#1\;]}

% CALIGRAPHY ALPHABET
% ===================
  \newcommand{\cA}{\mathcal{A}}
  \newcommand{\cB}{\mathcal{B}}
  \newcommand{\cC}{\mathcal{C}}
  \newcommand{\cD}{\mathcal{D}}
  \newcommand{\cE}{\mathcal{E}}
  \newcommand{\cF}{\mathcal{F}}
  \newcommand{\cG}{\mathcal{G}}
  \newcommand{\cI}{\mathcal{I}}
  \newcommand{\cJ}{\mathcal{J}}
  \newcommand{\cK}{\mathcal{K}}
  \newcommand{\cL}{\mathcal{L}}
  \newcommand{\cO}{\mathcal{O}}
  \newcommand{\cP}{\mathcal{P}}
  \newcommand{\cQ}{\mathcal{Q}}
  \newcommand{\cR}{\mathcal{R}}  
  \newcommand{\cS}{\mathcal{S}}
  \newcommand{\cV}{\mathcal{V}}
  \newcommand{\cW}{\mathcal{W}}


% ENERGY ALPHABET
% ===============
  \newcommand{\Vext}{V^{\mbox{\tiny ext}}}
  \newcommand{\Vint}{V^{\mbox{\tiny int}}}

% Second and Fourth Order Tensor ALPHABET
% ========================================
%\newcommand{\tC}{\mathbb{C}}
%\newcommand{\tI}{\mathbb{I}}
%\newcommand{\tV}{\mathbb{V}}
% fourth-order tensor 
\newcommand{\fotA}{\mathbb{A}}
% \newcommand{\fota}{{\rm a\kern-0.19em \tiny |}}
% \newcommand{\fota}{\mathbb{a}}
% \newcommand{\fota}{\UnicodeChar{2200}}
% \DeclareUnicodeCharacter{0177}{\fota}
% \newcommand{\fota}{\mathbb{a}}
%  \newcommand{\fota}{\Bbb{a}}
\newcommand{\fota}{\mathbbm{a}}
\newcommand{\fotC}{\mathbb{C}}
\newcommand{\fotc}{\mathbb{c}}

% \newcommand{\transpose}{^{\mbox{\scriptsize \sffamily T}}}
\newcommand{\transpose}{^{\mbox{\scriptsize \sffamily T}}}
% \newcommand{\negtranspose}{^{\mbox{\scriptsize \sffamily -T}}}
\newcommand{\minustranspose}{^{\mbox{\scriptsize \sffamily --T}}}
% \newcommand{\inverse}{^{\mbox{\scriptsize \sffamily \;-1}}}
% \newcommand{\inverse}{^{\mbox{\scriptsize \;-1}}}
\newcommand{\inverse}{^{-1}}

\newcommand{\tA}{\mathsf A}
%\newcommand{\ta}{\mbox{\sffamily a}}
%\newcommand{\tb}{\mbox{\sffamily b}}
%\newcommand{\tC}{\mbox{\sffamily C}}
%\newcommand{\tE}{\mbox{\sffamily E}}
%\newcommand{\te}{\mbox{\sffamily e}}
%\newcommand{\tF}{\mbox{\sffamily F}}
%\newcommand{\tI}{\mbox{\rmfamily I}}
%\newcommand{\tM}{\mbox{\sffamily M}}
%\newcommand{\tP}{\mbox{\sffamily P}}
%\newcommand{\tQ}{\mbox{\sffamily Q}}
%\newcommand{\tR}{\mbox{\sffamily R}}
%\newcommand{\ts}{\mbox{\sffamily s}}
%\newcommand{\tV}{\mbox{\sffamily V}}
\newcommand{\ta}{\mathsf a}
\newcommand{\tb}{\mathsf b}
\newcommand{\tC}{\mathsf C}
\newcommand{\tc}{\mbox{\sffamily\large c}}
\newcommand{\tICinv}{\tIbold_{C^{-1}}}
\newcommand{\tD}{\mathsf D}
\newcommand{\td}{\mathsf d}
\newcommand{\tE}{\mathsf E}
\newcommand{\te}{\mathsf e}
\newcommand{\tF}{\mathsf F}
\newcommand{\tI}{\mathsf I}
\newcommand{\ti}{\mbox{\rmfamily\large i}}
\newcommand{\tM}{\mathsf M}
\newcommand{\tN}{\mathsf N}
\newcommand{\tP}{\mathsf P}
\newcommand{\tQ}{\mathsf Q}
\newcommand{\tR}{\mathsf R}
\newcommand{\ts}{\mathsf s}
\newcommand{\tU}{\mathsf U}
\newcommand{\tV}{\mathsf V}
\newcommand{\tv}{\mbox{\sffamily\large v}}
\newcommand{\tw}{\mathsf w}

\newcommand{\tAbold}{\mbox{\sffamily\bfseries A}}
\newcommand{\tabold}{\mbox{\sffamily\bfseries a}}
\newcommand{\tBbold}{\mbox{\sffamily\bfseries B}}
\newcommand{\tbbold}{\mbox{\sffamily\bfseries b}}
\newcommand{\tCbold}{\mbox{\sffamily\bfseries C}}
\newcommand{\tcbold}{\mbox{\sffamily\bfseries\large c}}
\newcommand{\tDbold}{\mbox{\sffamily\bfseries D}}
\newcommand{\tdbold}{\mbox{\sffamily\bfseries d}}
\newcommand{\tEbold}{\mbox{\sffamily\bfseries E}}
\newcommand{\tebold}{\mbox{\sffamily\bfseries e}}
\newcommand{\tFbold}{\mbox{\sffamily\bfseries F}}
\newcommand{\tHbold}{\mbox{\sffamily\bfseries H}}
\newcommand{\thbold}{\mbox{\sffamily\bfseries h}}
\newcommand{\tIbold}{\mbox{\rmfamily\bfseries I}}
\newcommand{\tibold}{\mbox{\rmfamily\bfseries\large i}}
\newcommand{\tMbold}{\mbox{\sffamily\bfseries M}}
\newcommand{\tNbold}{\mbox{\sffamily\bfseries N}}
\newcommand{\tnbold}{\mbox{\sffamily\bfseries n}}
\newcommand{\tobold}{\mbox{\sffamily\bfseries o}}
\newcommand{\tPbold}{\mbox{\sffamily\bfseries P}}
\newcommand{\tQbold}{\mbox{\sffamily\bfseries Q}}
\newcommand{\tRbold}{\mbox{\sffamily\bfseries R}}
\newcommand{\tSbold}{\mbox{\sffamily\bfseries S}}
\newcommand{\tsbold}{\mbox{\sffamily\bfseries s}}
\newcommand{\tUbold}{\mbox{\sffamily\bfseries U}}
\newcommand{\tVbold}{\mbox{\sffamily\bfseries V}}
\newcommand{\tvbold}{\mbox{\sffamily\bfseries v}}
\newcommand{\tWbold}{\mbox{\sffamily\bfseries W}}
%\newcommand{\tvbold}{\mbox{\sffamily\bfseries\large v}}
\newcommand{\twbold}{\mbox{\sffamily\bfseries w}}
\newcommand{\tXbold}{\mbox{\sffamily\bfseries X}}
\newcommand{\txbold}{\mbox{\sffamily\bfseries x}}
\newcommand{\tYbold}{\mbox{\sffamily\bfseries Y}}
\newcommand{\tybold}{\mbox{\sffamily\bfseries y}}
\newcommand{\tZbold}{\mbox{\sffamily\bfseries Z}}
\newcommand{\tzbold}{\mbox{\sffamily\bfseries z}}





%\newcommand{\assembly}{\overset{n_{el}}{\underset{e=1}{
%  \mbox{ \sffamily\bfseries \large A } }}}

% Tensor INVARIANTS
% =================
\newcommand{\invCone}{\mbox{\rmfamily I}_C}
\newcommand{\invCtwo}{\mbox{\rmfamily II}_C}
\newcommand{\invCthree}{\mbox{\rmfamily III}_C}

% RIGID BODY  NOTATION
% ====================
% plain variable
\newcommand{\vucm}{\overset{*}{\ve{u}}}
\newcommand{\vvcm}{\overset{*}{\ve{v}}}
\newcommand{\vacm}{\overset{*}{\ve{a}}}
\newcommand{\vXcm}{\overset{*}{\ve{X}}}
\newcommand{\Xcm}{\overset{*}{X}}
\newcommand{\vxcm}{\overset{*}{\ve{x}}}
\newcommand{\vvarphicm}{\overset{*}{\ve{\varphi}}}
\newcommand{\vtheta}{\ve{\theta}}
\newcommand{\thetadot}{\dot{\theta}}
\newcommand{\vthetadot}{\dot{\ve{\theta}}}
\newcommand{\vthetaddot}{\ddot{\ve{\theta}}}
\newcommand{\vomega}{\ve{\omega}}
\newcommand{\vomegahat}{\widehat{\ve{\omega}}}
\newcommand{\vpsi}{\ve{\psi}}
\newcommand{\psidot}{\dot{\psi}}
\newcommand{\vpsidot}{\dot{\ve{\psi}}}
\newcommand{\valpha}{\ve{\alpha}}
\newcommand{\vpi}{\ve{\pi}}

\newcommand{\vvCinB}{}

%  at time n
\newcommand{\vucmn}{\overset{*}{\ve{u}}_n}
\newcommand{\vvcmn}{\overset{*}{\ve{v}}_n}
\newcommand{\vacmn}{\overset{*}{\ve{a}}_n}
\newcommand{\vXcmn}{\overset{*}{\ve{X}}_n}
\newcommand{\vxcmn}{\overset{*}{\ve{x}}_n}
\newcommand{\vvarphicmn}{\overset{*}{\ve{\varphi}}_n}
\newcommand{\vthetan}{\ve{\theta}_n}
\newcommand{\vthetadotn}{\dot{\ve{\theta}}_n}
\newcommand{\vthetaddotn}{\ddot{\ve{\theta}}_n}
\newcommand{\vomegan}{\ve{\omega}_n}
\newcommand{\valphan}{\ve{\alpha}_n}

% at time n+1
\newcommand{\vucmnp}{\overset{*}{\ve{u}}_{n+1}}
\newcommand{\vvcmnp}{\overset{*}{\ve{v}}_{n+1}}
\newcommand{\vacmnp}{\overset{*}{\ve{a}}_{n+1}}
\newcommand{\vXcmnp}{\overset{*}{\ve{X}}_{n+1}}
\newcommand{\vxcmnp}{\overset{*}{\ve{x}}_{n+1}}
\newcommand{\vvarphicmnp}{\overset{*}{\ve{\varphi}}_{n+1}}
\newcommand{\vthetanp}{\ve{\theta}_{n+1}}
\newcommand{\vthetadotnp}{\dot{\ve{\theta}}_{n+1}}
\newcommand{\vthetaddotnp}{\ddot{\ve{\theta}}_{n+1}}
\newcommand{\vomeganp}{\ve{\omega}_{n+1}}
\newcommand{\valphanp}{\ve{\alpha}_{n+1}}

% at time n+1 iteration i
\newcommand{\vucmnpi}{\overset{*}{\ve{u}}_{n+1}^{(i)}}
\newcommand{\vvcmnpi}{\overset{*}{\ve{v}}_{n+1}^{(i)}}
\newcommand{\vacmnpi}{\overset{*}{\ve{a}}_{n+1}^{(i)}}
\newcommand{\vXcmnpi}{\overset{*}{\ve{X}}_{n+1}^{(i)}}
\newcommand{\vxcmnpi}{\overset{*}{\ve{x}}_{n+1}^{(i)}}
\newcommand{\vvarphicmnpi}{\overset{*}{\ve{\varphi}}_{n+1}^{(i)}}
\newcommand{\Dvucmnpi}{\Delta \overset{*}{\ve{u}}_{n+1}^{(i)}}
\newcommand{\vthetanpi}{\ve{\theta}_{n+1}^{(i)}}
\newcommand{\vthetadotnpi}{\dot{\ve{\theta}}_{n+1}^{(i)}}
\newcommand{\vthetaddotnpi}{\ddot{\ve{\theta}}_{n+1}^{(i)}}
\newcommand{\vomeganpi}{\ve{\omega}_{n+1}^{(i)}}
\newcommand{\valphanpi}{\ve{\alpha}_{n+1}^{(i)}}


% at time n+1 iteration i+1
\newcommand{\vucmnpip}{\overset{*}{\ve{u}}_{n+1}^{(i+1)}}
\newcommand{\vvcmnpip}{\overset{*}{\ve{v}}_{n+1}^{(i+1)}}
\newcommand{\vacmnpip}{\overset{*}{\ve{a}}_{n+1}^{(i+1)}}
\newcommand{\vXcmnpip}{\overset{*}{\ve{X}}_{n+1}^{(i+1)}}
\newcommand{\vxcmnpip}{\overset{*}{\ve{x}}_{n+1}^{(i+1)}}
\newcommand{\vvarphicmnpip}{\overset{*}{\ve{\varphi}}_{n+1}^{(i+1)}}
\newcommand{\vthetanpip}{\ve{\theta}_{n+1}^{(i+1)}}
\newcommand{\vthetadotnpip}{\dot{\ve{\theta}}_{n+1}^{(i+1)}}
\newcommand{\vthetaddotnpip}{\ddot{\ve{\theta}}_{n+1}^{(i+1)}}
\newcommand{\vomeganpip}{\ve{\omega}_{n+1}^{(i+1)}}
\newcommand{\valphanpip}{\ve{\alpha}_{n+1}^{(i+1)}}


% volumes and other geometry
\newcommand{\OmegaR}{\Omega_R}
\newcommand{\intOmegaR}{\int_{\Omega_R}}
\newcommand{\intOmegaRe}{\int_{\Omega_R^e}}
\newcommand{\OmegaOne}{\Omega_1}
\newcommand{\OmegaTwo}{\Omega_2}

% masses and ineritas
%\newcommand{\inertia}{{\mbox{\rmfamily\bfseries I}}}
%\newcommand{\inertiaJ}{{\mbox{\rmfamily\bfseries J}}}
\newcommand{\inertia}{\mathbb I}
\newcommand{\inertiaJ}{\mathbb J}

% inertia at time n
\newcommand{\inertian}{{\mathbb I}_n}
\newcommand{\inertiaJn}{{\mathbb J}_n}

% END RIGID BODY NOTATION
% =======================


% *---------------*
% | BASIS VECTORS |
% *---------------*----------------------------------------------

%\newcommand{\veaa}{\hat{\ve{e}}_{01}}
%\newcommand{\veab}{\hat{\ve{e}}_{02}}
%\newcommand{\veac}{\hat{\ve{e}}_{03}}
%\newcommand{\veaa}{^{A}\hat{\ve{e}}_{1}}
%\newcommand{\veab}{^{A}\hat{\ve{e}}_{2}}
%\newcommand{\veac}{^{A}\hat{\ve{e}}_{3}}
%\newcommand{\veaa}{\hat{\ve{a}}_{1}}
%\newcommand{\veab}{\hat{\ve{a}}_{2}}
%\newcommand{\veac}{\hat{\ve{a}}_{3}}
\newcommand{\vea}{\hat{\ve{\sf{a}}}}
\newcommand{\veaa}{\hat{\ve{\sf{a}}}_1}
\newcommand{\veab}{\hat{\ve{\sf{a}}}_2}
\newcommand{\veac}{\hat{\ve{\sf{a}}}_3}
%
%\newcommand{\veba}{\hat{\ve{e}}_{11}}
%\newcommand{\vebb}{\hat{\ve{e}}_{12}}
%\newcommand{\vebc}{\hat{\ve{e}}_{13}}
%\newcommand{\veba}{\hat{\ve{b}}_{1}}
%\newcommand{\vebb}{\hat{\ve{b}}_{2}}
%\newcommand{\vebc}{\hat{\ve{b}}_{3}}
% \newcommand{\veb}{\hat{\ve{\sf{b}}}}
% \newcommand{\veba}{\hat{\ve{\sf{b}}}_1}
% \newcommand{\vebb}{\hat{\ve{\sf{b}}}_2}
% \newcommand{\vebc}{\hat{\ve{\sf{b}}}_3}
\newcommand{\vebhat}{\hat{\mathbf{b}}}
\newcommand{\vebIhat}{\hat{\mathbf{b}}_I}
\newcommand{\vebihat}{\hat{\mathbf{b}}_i}
\newcommand{\vebahat}{\hat{\mathbf{b}}_1}
\newcommand{\vebbhat}{\hat{\mathbf{b}}_2}
\newcommand{\vebchat}{\hat{\mathbf{b}}_3}
\newcommand{\vebonehat}{\hat{\mathbf{b}}_1}
\newcommand{\vebtwohat}{\hat{\mathbf{b}}_2}
\newcommand{\vebthreehat}{\hat{\mathbf{b}}_3}
%

%\newcommand{\veca}{\hat{\ve{e}}_{21}}
%\newcommand{\vecb}{\hat{\ve{e}}_{22}}
%\newcommand{\vecc}{\hat{\ve{e}}_{23}}
\newcommand{\veca}{\hat{\ve{c}}_{1}}
\newcommand{\vecb}{\hat{\ve{c}}_{2}}
\newcommand{\vecc}{\hat{\ve{c}}_{3}}
%%
\newcommand{\veda}{\hat{\ve{e}}_{31}}
\newcommand{\vedb}{\hat{\ve{e}}_{32}}
\newcommand{\vedc}{\hat{\ve{e}}_{33}}
%
\newcommand{\veea}{\hat{\ve{e}}_{41}}
\newcommand{\veeb}{\hat{\ve{e}}_{42}}
\newcommand{\veec}{\hat{\ve{e}}_{43}}
%
%\newcommand{\veca}{\hat{\ve{e}}_{\cC1}}
%\newcommand{\vecb}{\hat{\ve{e}}_{\cC2}}
%\newcommand{\vecc}{\hat{\ve{e}}_{\cC3}}
%
%\newcommand{\vefa}{\hat{\ve{e}}_{\cF1}}
%\newcommand{\vefb}{\hat{\ve{e}}_{\cF2}}
%\newcommand{\vefc}{\hat{\ve{e}}_{\cF3}}
\newcommand{\vefa}{\hat{\ve{\sf{f}}}_1}
\newcommand{\vefb}{\hat{\ve{\sf{f}}}_2}
\newcommand{\vefc}{\hat{\ve{\sf{f}}}_3}

%
\newcommand{\veva}{\hat{\ve{e}}_{\cV1}}
\newcommand{\vevb}{\hat{\ve{e}}_{\cV2}}
\newcommand{\vevc}{\hat{\ve{e}}_{\cV3}}

\newcommand{\veihat}{\hat{\ve{\sf{i}}}}
\newcommand{\vejhat}{\hat{\ve{\sf{j}}}}
\newcommand{\vekhat}{\hat{\ve{\sf{k}}}}

% \newcommand{\veEihat}{\hat{\ve{E}}_I}
% \newcommand{\veEjhat}{\hat{\ve{E}}_J}
% \newcommand{\veEkhat}{\hat{\ve{E}}_K}
% \newcommand{\veElhat}{\hat{\ve{E}}_L}

\newcommand{\veEihat}{\hat{\mathbf{E}}_I}
\newcommand{\veEjhat}{\hat{\mathbf{E}}_J}
\newcommand{\veEkhat}{\hat{\mathbf{E}}_K}
\newcommand{\veElhat}{\hat{\mathbf{E}}_L}

% \newcommand{\veeihat}{\hat{\ve{e}}_i}
% \newcommand{\veejhat}{\hat{\ve{e}}_j}
% \newcommand{\veekhat}{\hat{\ve{e}}_k}
% \newcommand{\veelhat}{\hat{\ve{e}}_l}

\newcommand{\veeihat}{\hat{\mathbf{e}}_i}
\newcommand{\veejhat}{\hat{\mathbf{e}}_j}
\newcommand{\veekhat}{\hat{\mathbf{e}}_k}
\newcommand{\veelhat}{\hat{\mathbf{e}}_l}

% \newcommand{\veeonehat}{\hat{\ve{\sf{e}}}_1}
% \newcommand{\veetwohat}{\hat{\ve{\sf{e}}}_2}
% \newcommand{\veethreehat}{\hat{\ve{\sf{e}}}_3}
% \newcommand{\veeonehat}{\hat{\ve{e}}_1}
% \newcommand{\veetwohat}{\hat{\ve{e}}_2}
% \newcommand{\veethreehat}{\hat{\ve{e}}_3}
 
\newcommand{\veeonehat}{\hat{\mathbf{e}}_1}
\newcommand{\veetwohat}{\hat{\mathbf{e}}_2}
\newcommand{\veethreehat}{\hat{\mathbf{e}}_3}


\newcommand{\vEsubI}{\mathbf{E}_I}
\newcommand{\vEsubJ}{\mathbf{E}_J}
\newcommand{\vEsubK}{\mathbf{E}_K}
\newcommand{\vEone}{\mathbf{E}_1}
\newcommand{\vEtwo}{\mathbf{E}_2}
\newcommand{\vEthree}{\mathbf{E}_3}
\newcommand{\vEsubalpha}{\mathbf{E}_{\alpha}}
\newcommand{\vEsubbeta}{\mathbf{E}_{\beta}}

\newcommand{\vesubi}{\mathbf{e}_i}
\newcommand{\vesubj}{\mathbf{e}_j}
\newcommand{\vesubk}{\mathbf{e}_k}
\newcommand{\veone}{\mathbf{e}_1}
\newcommand{\vetwo}{\mathbf{e}_2}
\newcommand{\vethree}{\mathbf{e}_3}
\newcommand{\vesubalpha}{\mathbf{e}_{\alpha}}
\newcommand{\vesubbeta}{\mathbf{e}_{\beta}}


% FEA OPERATOR NOTATION
% =====================
\newcommand{\Moper}{\mathcal{M}(\cdot,\cdot)}
\newcommand{\Meoper}{\mathcal{M}(\cdot,\cdot)^e}
\newcommand{\Noper}{\mathcal{N}(\cdot;\cdot,\cdot)}
\newcommand{\Neoper}{\mathcal{N}(\cdot;\cdot,\cdot)^e}
\newcommand{\Boper}{(\cdot,\cdot)}
\newcommand{\Beoper}{(\cdot,\cdot)^e}
\newcommand{\Toper}{(\cdot,\cdot)_{\Gamma_H}}
\newcommand{\Teoper}{(\cdot,\cdot)^e_{\Gamma_H}}

% FEA INTEGERS
% ============
\newcommand{\nel}{n_{\mbox{\tiny el}}}
\newcommand{\nen}{n_{\mbox{\tiny en}}}
\newcommand{\nsd}{n_{\mbox{\tiny sd}}}
%\newcommand{\neq}{n_{\mbox{\tiny eq}}}
\newcommand{\numeq}{n_{\mbox{\tiny eq}}}
\newcommand{\nts}{n_{\mbox{\tiny ts}}}
\newcommand{\naf}{n_{\mbox{\tiny af}}}
\newcommand{\nam}{n_{\mbox{\tiny am}}}
\newcommand{\nat}{n_{\mbox{\tiny at}}}

\newcommand{\GDYN}{G_{\mbox{\tiny DYN}}}

% SHAPE FUNCTION TRIAL FUNCTION
% =============================

\newcommand{\Uh}{U^h}
\newcommand{\Uhi}{U^h_i}
\newcommand{\vUh}{\ve{U}^h}
\newcommand{\vUt}{\ve{U}_t}
\newcommand{\vUhdot}{\dot{\ve{U}}^h}

\newcommand{\Wh}{W^h}
\newcommand{\Whi}{W^h_i}
\newcommand{\vWh}{\ve{W}^h}

\newcommand{\vTt}{\ve{T}_t}

\newcommand{\Kh}{K^h}
\newcommand{\Khi}{K^h_i}
\newcommand{\vKh}{\ve{K}^h}

\newcommand{\vcK}{\ve{\mathcal{K}}}
\newcommand{\vcKh}{\ve{\mathcal{K}}^h}
\newcommand{\vcV}{\ve{\mathcal{V}}}
\newcommand{\vcVh}{\ve{\mathcal{V}}^h}
\newcommand{\vcW}{\ve{\mathcal{W}}}
\newcommand{\vcWh}{\ve{\mathcal{W}}^h}
\newcommand{\vcS}{\ve{\mathcal{S}}}
\newcommand{\vcSh}{\ve{\mathcal{S}}^h}
\newcommand{\vcC}{\ve{\mathcal{C}}}

% RIGID BODY MASS CENTERS
% =======================
\newcommand{\Bstar}{B^{*}}

% INTEGRATION SYMBOLS
% ===================

\newcommand{\Omegae}{\Omega^e}

\newcommand{\intBox}{\int_{\Box}}

\newcommand{\intOmega}{\int_{\Omega}}
\newcommand{\intOmegae}{\int_{\Omega^e}}
\newcommand{\intomega}{\int_{\omega}}
\newcommand{\intomegae}{\int_{\omega^e}}


\newcommand{\intGamma}{\int_{\Gamma}}
\newcommand{\intgamma}{\int_{\gamma}}
\newcommand{\intGammaH}{\int_{\Gamma_H}}
\newcommand{\intGammaT}{\int_{\Gamma_T}}
\newcommand{\intGammaU}{\int_{\Gamma_U}}
\newcommand{\intGammaC}{\int_{\Gamma_C}}
\newcommand{\intGammaHe}{\int_{\Gamma^e_H}}
\newcommand{\intgammah}{\int_{\gamma_h}}

\newcommand{\intOmegaalpha}{\int_{\Omega^{(\alpha)}}}
\newcommand{\intGammaalpha}{\int_{\Gamma^{(\alpha)}}}
\newcommand{\intGammaone}{\int_{\Gamma^{(1)}}}
\newcommand{\intGammatwo}{\int_{\Gamma^{(2)}}}
\newcommand{\intGammaTalpha}{\int_{\Gamma_T^{(\alpha)}}}
\newcommand{\intGammaCalpha}{\int_{\Gamma_C^{(\alpha)}}}
\newcommand{\intGammaCone}{\int_{\Gamma_C^{(1)}}}
\newcommand{\intGammaCtwo}{\int_{\Gamma_C^{(2)}}}

\newcommand{\GammaH}{\Gamma_{H}}


\newcommand{\dOmega}{\; d\Omega}
\newcommand{\domega}{\; d\omega}
\newcommand{\dGamma}{\; d\Gamma}
\newcommand{\dgamma}{\; d\gamma}
\newcommand{\dGammaC}{\; d\Gamma_C}
\newcommand{\dBox}{\; d\;\Box}
\newcommand{\dOmegaalpha}{\; d\Omega^{(\alpha)}} 
\newcommand{\dGammaalpha}{\; d\Gamma^{(\alpha)}}

\newcommand{\deltae}{\delta^{\mbox{\tiny e}}}
\newcommand{\vdeltae}{\ve{\delta}^{\mbox{\tiny e}}}
\newcommand{\Deltae}{\Delta^{\mbox{\tiny e}}}
\newcommand{\vDeltae}{\ve{\Delta}^{\mbox{\tiny e}}}
 
\newcommand{\sone}{^{(1)}}
\newcommand{\stwo}{^{(2)}}
\newcommand{\sa}{^{(\alpha)}}


% *-----------------------------*
% | STRUCTURAL DYNAMICS SYMBOLS |
% *-----------------------------*--------------------------------

%\newcommand{\vd}{\ve{d}}
%\newcommand{\vv}{\ve{v}}
%\newcommand{\va}{\ve{a}}

\newcommand{\vdn}{\ve{d}_{n}}
\newcommand{\vun}{\ve{u}_{n}}
\newcommand{\vue}{\ve{u}^{\mbox{\tiny e}}}
\newcommand{\vUn}{\ve{U}_{n}}
\newcommand{\vvn}{\ve{v}_{n}}
\newcommand{\van}{\ve{a}_{n}}

\newcommand{\vdnp}{\ve{d}_{n+1}}
\newcommand{\vunp}{\ve{u}_{n+1}}
\newcommand{\vUnp}{\ve{U}_{n+1}}
\newcommand{\vvnp}{\ve{v}_{n+1}}
\newcommand{\vanp}{\ve{a}_{n+1}}

\newcommand{\vdnpa}{\ve{d}_{n+\alpha}}
\newcommand{\vunpa}{\ve{u}_{n+\alpha}}
\newcommand{\vUnpa}{\ve{U}_{n+\alpha}}
\newcommand{\vvnpa}{\ve{v}_{n+\alpha}}
\newcommand{\vVnpa}{\ve{V}_{n+\alpha}}
\newcommand{\vanpa}{\ve{a}_{n+\alpha}}
\newcommand{\vAnpa}{\ve{A}_{n+\alpha}}


\newcommand{\vdnm}{\ve{d}_{n-1}}
\newcommand{\vunm}{\ve{u}_{n-1}}
\newcommand{\vvnm}{\ve{v}_{n-1}}
\newcommand{\vanm}{\ve{a}_{n-1}}

\newcommand{\vdni}{\ve{d}_{n}^{(i)}}
\newcommand{\vuni}{\ve{u}_{n}^{(i)}}
\newcommand{\vUni}{\ve{U}_{n}^{(i)}}
\newcommand{\vvni}{\ve{v}_{n}^{(i)}}
\newcommand{\vani}{\ve{a}_{n}^{(i)}}

\newcommand{\vdnpi}{\ve{d}_{n+1}^{(i)}}
\newcommand{\vunpi}{\ve{u}_{n+1}^{(i)}}
\newcommand{\vUnpi}{\ve{U}_{n+1}^{(i)}}
\newcommand{\vvnpi}{\ve{v}_{n+1}^{(i)}}
\newcommand{\vanpi}{\ve{a}_{n+1}^{(i)}}

\newcommand{\vdnpip}{\ve{d}_{n+1}^{(i+1)}}
\newcommand{\vunpip}{\ve{u}_{n+1}^{(i+1)}}
\newcommand{\vVnpip}{\ve{U}_{n+1}^{(i+1)}}
\newcommand{\vvnpip}{\ve{v}_{n+1}^{(i+1)}}
\newcommand{\vanpip}{\ve{a}_{n+1}^{(i+1)}}

%\newcommand{\vM}{\ve{M}}
\newcommand{\vMs}{\ve{M}^{*}}
\newcommand{\vMnpi}{\vM_{n+1}^{(i)}}
%\newcommand{\vC}{\ve{C}}
\newcommand{\vDnpi}{\vD_{n+1}^{(i)}}
%\newcommand{\vK}{\ve{K}}
\newcommand{\vKe}{\ve{K}^{\mbox{\tiny e}}}
\newcommand{\vKd}{\ve{K}(\ve{d})}
\newcommand{\vKnpi}{\ve{K}_{n+1}^{(i)}}
%\newcommand{\vN}{\ve{N}}
\newcommand{\vNd}{\ve{N}(\ve{d})}
\newcommand{\vNdn}{\ve{N}(\ve{d}_{n})}
\newcommand{\vNdnp}{\ve{N}(\ve{d}_{n+1})}
\newcommand{\vNdnpi}{\ve{N}(\ve{d}_{n+1}^{(i)})}
\newcommand{\vNdnpip}{\ve{N}(\ve{d}_{n+1}^{(i+1)})}

\newcommand{\vVn}{\ve{V}_{n}}
\newcommand{\vVnp}{\ve{V}_{n+1}}

\newcommand{\vvarphin}{\ve{\varphi}_{n}}
\newcommand{\vvarphinp}{\ve{\varphi}_{n+1}}
\newcommand{\vvarphinpa}{\ve{\varphi}_{n+\alpha}}


\newcommand{\vFint}{\ve{F}^{\mbox{\tiny int}}}
\newcommand{\vFinte}{\ve{F}^{\mbox{\tiny int,e}}}
\newcommand{\vFintvisc}{\ve{F}^{\mbox{\tiny int,visc}}}
\newcommand{\vFintelas}{\ve{F}^{\mbox{\tiny int,elas}}}
\newcommand{\vFiner}{\ve{F}^{\mbox{\tiny iner}}}
\newcommand{\vFext}{\ve{F}^{\mbox{\tiny ext}}}
\newcommand{\vFexte}{\ve{F}^{\mbox{\tiny ext,e}}}
\newcommand{\vFextn}{\vFext(t_{n})}
\newcommand{\vFextnp}{\vFext(t_{n+1})}
\newcommand{\vFextnpa}{\vFext(t_{n+1+\alpha})}
\newcommand{\vFapp}{\ve{F}^{\mbox{\tiny app}}}

\newcommand{\vMext}{\ve{M}^{ext}}
\newcommand{\vMapp}{\ve{M}^{\mbox{\tiny app}}}
\newcommand{\vTapp}{\ve{T}^{\mbox{\tiny app}}}



\newcommand{\vfint}{\ve{f}^{\mbox{\tiny int}}}
\newcommand{\vfext}{\ve{f}^{\mbox{\tiny ext}}}
\newcommand{\vfextn}{\vfext(t_{n})}
\newcommand{\vfextnp}{\vfext(t_{n+1})}
\newcommand{\vfextnpa}{\vfext(t_{n+1+\alpha})}

\newcommand{\vddd}{\ddot{\ve{d}}}
\newcommand{\vdd}{\dot{\ve{d}}}
% \newcommand{\vd}{\ve{d}}  % already defined above


\newcommand{\pvd}{\tilde{\ve{d}}}
\newcommand{\pvu}{\tilde{\ve{u}}}
\newcommand{\pvv}{\tilde{\ve{v}}}
\newcommand{\pva}{\tilde{\ve{a}}}
\newcommand{\pvvarphi}{\tilde{\ve{\varphi}}}

\newcommand{\pvdn}{\tilde{\ve{d}}_{n}}
\newcommand{\pvun}{\tilde{\ve{u}}_{n}}
\newcommand{\pvvn}{\tilde{\ve{v}}_{n}}
\newcommand{\pvvarphin}{\tilde{\ve{\varphi}}_{n}}

% time step n+1
\newcommand{\pvdnp}{\tilde{\ve{d}}_{n+1}}
\newcommand{\pvunp}{\tilde{\ve{u}}_{n+1}}
\newcommand{\pvvnp}{\tilde{\ve{v}}_{n+1}}
\newcommand{\pvvcmnp}{\overset{*}{\tilde{\ve{v}}}_{n+1}}
\newcommand{\pvacmnp}{\overset{*}{\tilde{\ve{a}}}_{n+1}}
\newcommand{\pvvarphinp}{\tilde{\ve{\varphi}}_{n+1}}
\newcommand{\pvvarphicmnp}{\overset{*}{\tilde{\ve{\varphi}}}_{n+1}}
\newcommand{\pvthetanp}{\tilde{\ve{\theta}}_{n+1}}
\newcommand{\pvomeganp}{\tilde{\ve{\omega}}_{n+1}}
\newcommand{\pvalphanp}{\tilde{\ve{\alpha}}_{n+1}}

% time step n-1
\newcommand{\pvdnm}{\tilde{\ve{d}}_{n-1}}
\newcommand{\pvunm}{\tilde{\ve{u}}_{n-1}}
\newcommand{\pvvnm}{\tilde{\ve{v}}_{n-1}}


% time step n+1 iteration i
\newcommand{\pvdnpi}{\tilde{\ve{d}}_{n+1}^{(i)}}
\newcommand{\pvunpi}{\tilde{\ve{u}}_{n+1}^{(i)}}
\newcommand{\pvvnpi}{\tilde{\ve{v}}_{n+1}^{(i)}}
\newcommand{\pvvarphinpi}{\tilde{\ve{\varphi}}_{n+1}^{(i)}}
\newcommand{\pvvarphicmnpi}{\overset{*}{\tilde{\ve{\varphi}}}_{n+1}^{(i)}}
\newcommand{\pvthetanpi}{\tilde{\ve{\theta}}_{n+1}^{(i)}}
\newcommand{\pvomeganpi}{\tilde{\ve{\omega}}_{n+1}^{(i)}}
\newcommand{\pvalphanpi}{\tilde{\ve{\alpha}}_{n+1}^{(i)}}


\newcommand{\pvqnp}{\tilde{\vq}_{n+1}}
\newcommand{\pvqdotnp}{\tilde{\vqdot}_{n+1}}
\newcommand{\pvqddotnp}{\tilde{\vqddot}_{n+1}}
\newcommand{\vqn}{\vq_n}
\newcommand{\vqnp}{\vq_{n+1}}
\newcommand{\vqdotn}{\vqdot_n}
\newcommand{\vqdotnp}{\vqdot_{n+1}}
\newcommand{\vqddotn}{\vqddot_n}
\newcommand{\vqddotnp}{\vqddot_{n+1}}
\newcommand{\vqnpip}{\vq_{n+1}^{(i+1)}}
\newcommand{\vqnpi}{\vq_{n+1}^{(i)}}
\newcommand{\vqdotnpip}{\vqdot_{n+1}^{(i+1)}}
\newcommand{\vqdotnpi}{\vqdot_{n+1}^{(i)}}
\newcommand{\vqddotnpip}{\vqddot_{n+1}^{(i+1)}}
\newcommand{\vqddotnpi}{\vqddot_{n+1}^{(i)}}


\newcommand{\Dt}{\Delta t}
\newcommand{\DU}{\Delta U}
\newcommand{\DUdot}{\Delta \dot{U}}
% \newcommand{\DvU}{\Delta \ve{U}}
\newcommand{\DvUdot}{\Delta \dot{\vU}}

\newcommand{\Dvdnp}{\Delta \ve{d}_{n+1}}
\newcommand{\Dvunp}{\Delta \ve{u}_{n+1}}
\newcommand{\Dvvnp}{\Delta \ve{v}_{n+1}}
\newcommand{\Dvanp}{\Delta \ve{a}_{n+1}}



\newcommand{\Dvai}{\Delta \ve{a}^{(i)}}



\newcommand{\Dvdnpi}{\Delta \ve{d}_{n+1}^{(i)}}
\newcommand{\Dvunpi}{\Delta \ve{u}_{n+1}^{(i)}}
\newcommand{\Dvvnpi}{\Delta \ve{v}_{n+1}^{(i)}}
\newcommand{\Dvanpi}{\Delta \ve{a}_{n+1}^{(i)}}


\newcommand{\vRn}{\ve{R}_{n}}
\newcommand{\vRnp}{\ve{R}_{n+1}}
\newcommand{\vRnpa}{\ve{R}_{n+1+\alpha}}
\newcommand{\vRnpi}{\ve{R}_{n+1}^{(i)}}
\newcommand{\vRnpip}{\vR_{n+1}^{(i+1)}} 
% *--------------------------------------------------------------
 



% *------------------------*
% | FINITE ELEMENT SYMBOLS |
% *------------------------*-------------------------------------

\newcommand{\oldassembly}{\sum_{e=1}^{\nel}}
\newcommand{\assembly}{\overset{\nel}{\underset{e=1}{
  \mbox{ \sffamily\bfseries \large A } }}}

% *--------------------------------------------------------------
 




% *--------------------------*
% | MATRIX NOTATIONS/SPACES  |
% *--------------------------*-----------------------------------

\newcommand{\Mnsdplus}{\mathbb{M}^{\nsd}_{\; +}}
\newcommand{\Moneplus}{\mathbb{M}^{1}_{\; +}}
\newcommand{\Mtwoplus}{\mathbb{M}^{2}_{\; +}}
\newcommand{\Mthreeplus}{\mathbb{M}^{3}_{\; +}}

% *--------------------------*
% | REAL AND COMPLEX NUMBERS |
% *--------------------------*-----------------------------------

 \newcommand{\real}{\mathbb{R}}
 \newcommand{\realone}{\mathbb{R}^1}
 \newcommand{\realtwo}{\mathbb{R}^2}
 \newcommand{\realthree}{\mathbb{R}^3}
 \newcommand{\realn}{\mathbb{R}^n}
 \newcommand{\realnp}{\mathbb{R}^{n+1}}
 \newcommand{\realm}{\mathbb{R}^m}
 \newcommand{\realneq}{\mathbb{R}^{\numeq}}
 \newcommand{\realnsd}{\mathbb{R}^{\nsd}}
 \newcommand{\nqs}{{n_{qs}}}
 \newcommand{\realnqs}{\mathbb{R}^{\nqs}}
 \newcommand{\realplus}{{\mathbb{R}^+}}
 \newcommand{\realminus}{{\mathbb{R}^-}}
 \newcommand{\realpluscone}{{\mathbb{R}^+_{\cC}}}
 \newcommand{\realminuscone}{{\mathbb{R}^-_{\cC}}}

 \newcommand{\complex}{\mathbb{C}}
 \newcommand{\complexone}{\mathbb{C}^1}
 \newcommand{\complextwo}{\mathbb{C}^2}
 \newcommand{\complexthree}{\mathbb{C}^3}
 \newcommand{\complexn}{\mathbb{C}^n}

% *-----------------*
% | SET OF INTEGERS |
% *-----------------*--------------------------------------------
 \newcommand{\integer}{\mathbb{J}}

% *------------------*
% | SETS OF MATRICES |
% *------------------*-------------------------------------------

% \newcommand{\sym}{\mathbb{S}}
 \newcommand{\sympd}{\mathbb{S}_{>}}
 \newcommand{\sympsd}{\mathbb{S}_{\geq}}

% *--------------------------------------------------------------




% *----------------*
% | SCRIPT SYMBOLS |
% *----------------*---------------------------------------------

%\DeclareFontFamily{U}{callig}{}
%\DeclareFontShape{U}{callig}{m}{n}{<5> <6> <7> <8> <9> <10> <10.95> <12> <14.4>
%<17.28> callig15}{}
%\DeclareSymbolFont{calligr}{U}{callig}{m}{n}
%\DeclareMathSymbol{\cCr}{\mathalpha}{calligr}{``43} % curve
%\DeclareMathSymbol{\cEp}{\mathalpha}{calligr}{``45} % space
%\DeclareMathSymbol{\cFi}{\mathalpha}{calligr}{``46} % Rayleigh dissipation
%\DeclareMathSymbol{\cLg}{\mathalpha}{calligr}{``4C} % Lagrangian
%\DeclareMathSymbol{\cPt}{\mathalpha}{calligr}{``50} % point
%\DeclareMathSymbol{\cQn}{\mathalpha}{calligr}{``51} % configuration space
%\DeclareMathSymbol{\cRf}{\mathalpha}{calligr}{``52} % reference frame
%\DeclareMathSymbol{\cSs}{\mathalpha}{calligr}{``53} % dynamical system
%\DeclareMathSymbol{\cOi}{\mathalpha}{calligr}{``4F} % origin of ref. frame
%\DeclareMathSymbol{\cVc}{\mathalpha}{calligr}{``56} % vector space

%\newcommand{\cCur}{\cCr\,}
%\newcommand{\cEsp}{\cEp\,}
%\newcommand{\cFdi}{\cFi\,}
%\newcommand{\cPnt}{\cPt\,}
%\newcommand{\cLag}{\cLg\,}
%\newcommand{\cQon}{\cQn\,}
%\newcommand{\cRef}{\cRf\,}
%\newcommand{\cSys}{\cSs\,}
%\newcommand{\cOri}{\cOi\,}
%\newcommand{\cVec}{\cVc\,}

%\newcommand{\cP}[1]{\ensuremath{\cPnt\;_{#1}}}

% *--------------------------------------------------------------




% *---------------------*
% | PAGE SPECIFICATIONS |
% *---------------------*----------------------------------------
% 

%  theoremstyle{plain}
%  \theorembodyfont{\rmfamily}
%  \newtheorem{assumption}{Assumption}[section]
%  \newtheorem{implication}{Implication}[section]
%  \newtheorem{justification}{Justification}[section]
%%  \newtheorem{proof}{Proof}[section]
%  \newtheorem{prooftwo}{Proof}[section]
%  \newtheorem{remark}{Remark}[section]
  \newtheorem{result}{Result}[section]
%  \newtheorem{definition}{Definition}[section]
%  \newtheorem{example}{Example}[section]

% *----------------------------*
% | HOVEY'S Hproof ENVIRONMENT |
% *----------------------------*----------------------------------

% The Hproof environment needs the package amssymb, so load
% is here and now, just in case it has not previously been
% loaded 

%\usepackage{amssymb}  % must have been included already
\definecolor{hdarkblue}{rgb}{0,0,0.5}  % dark blue [0,1] channel
% \definecolor{hindigo}{rgb}{0.294,0,0.510}  % #4b0082 
\definecolor{hdimdimgray}{HTML}{262626}  
\definecolor{hindigo}{HTML}{4B0082}  % #4b0082 
\definecolor{hdarkgray}{rgb}{0.05,0.05,0.05}
\definecolor{hlightgray}{rgb}{0.95,0.95,0.95}
\newenvironment{Hproof}{           % beginning the environment
    %\\
    \medskip
%    \begin{center}
%    \begin{tabular}{|l}
%    \begin{minipage}[b]{1.00\linewidth}
    \sffamily\footnotesize
    \noindent\makebox[\linewidth]{\color{hdarkblue}\rule{\linewidth}{0.4pt}}
    %\color{darkgray}\noindent{\em Proof.} \\
    \color{hdarkblue} \noindent{\em Proof.} \\
    %\color{blue4} \noindent{\em Proof.} \\  % blue4 does not work
     %%% https://tex.stackexchange.com/questions/236949/how-to-change-the-background-color-within-a-page
     %%%\begingroup
     %%%\offinterlineskip
     %%%\hbox to 0pt{%
     %%%\kern-\paperwidth
     %%%\vtop to 0pt{%
     %%%\color{red}%
     %%%\hrule width 2\paperwidth height \paperheight
     %%%\vss
     %%%} % vtop 
     %%%\hss
     %%%} % hbox
     %%%\endgroup
     %%%
     %\smallskip 
               } {              % ending the environment
    \hfill $\blacksquare$
    %\noindent\makebox[\linewidth]{\rule{\linewidth}{0.4pt}}
    %\noindent\makebox[\linewidth]{\color{blue}\rule{\linewidth}{4.4pt}}
    \noindent\makebox[\linewidth]{\color{hdarkblue}\rule{\linewidth}{0.4pt}}
%     \end{minipage} 
%    \end{tabular}
%    \end{center}
    \medskip
    %\\
    }               % end definition of Hproof new environment

% *-----------------------------*
% | HOVEY'S Hresult ENVIRONMENT |
% *-----------------------------*---------------------------------

% This newenvironment working hinges on the definition a priori
% or the newtheorem result

\newenvironment{Hresult}{        % beginning the environment
    %\\
    \medskip
    \begin{center}
    %\begin{tabular}{|l|} \hline
    \begin{tabular}{|l} % \hline
    \begin{minipage}[b]{1.00\linewidth}
      % \medskip
      \begin{result}
      \em \smallskip       }{               % ending the environment
      \end{result}
    \end{minipage}
    \\ % \hline
    \end{tabular}
    \end{center}
    % \medskip
    %\\
    }               % end definition of Hresult new environment


% *------------------------------*
% | HOVEY'S Hexample ENVIRONMENT |
% *------------------------------*--------------------------------

\newcounter{Hexamplectr}
\renewcommand{\theHexamplectr}{\arabic{Hexamplectr}.}
\newenvironment{Hexample}[1]
 {
  \stepcounter{Hexamplectr}%
  \medskip
  %\begin{center}
    %\fbox{\parbox[b]{0.9\linewidth}{
      %\noindent\makebox[\linewidth]{\color{hlightgray}\rule{\linewidth}{4.4pt}}
      %\noindent {\colorbox{hlightgray}{\parbox{\linewidth}{\bf \center Example \theHexamplectr}}} \\ \\ #1
      \noindent {\colorbox{hlightgray}{\parbox{\linewidth}{\bf Example \theHexamplectr}}} \\ #1 % \\ \centering $\square$
      %\noindent {\colorbox{hlightgray}{\parbox{\linewidth}{\bf Example \theHexamplectr #1 }}}  % don't use b/c does not span 2 or more pages
      %\noindent \colorbox{red}{test}{\bf Example \theHexamplectr} #1
      % \noindent {\bf Example \theHexamplectr} #1
%      \noindent {\bf Example \theHexamplectr} {\colorbox{red}{#1}}
                                   %}
         %}
  %\end{center}
      %\noindent {\colorbox{hlightgray}{\parbox{\linewidth}{\bf \centering $\cdots$}}} \vspace{2cm}
      %\noindent\makebox[\linewidth]{\color{hlightgray}\rule{\linewidth}{12pt}}
 }{
  \medskip
 }
 
%\newenvironment{Hexample}[1]{
%    \begin{center}
%    \fbox{\parbox[b]{\linewidth}{#1}}
%    \end{center}
%    }{
%    }

% *---------------------------*
% | HOVEY'S Hexpo ENVIRONMENT |
% *---------------------------*--------------------------------

\newcounter{Hexpoctr}
\renewcommand{\theHexpoctr}{\arabic{Hexpoctr}.}
\newenvironment{Hexpo}[1]  % takes 1 argument
{
  \stepcounter{Hexpoctr}%
  \medskip
  \noindent 
  %{\colorbox{hlightgray}{\parbox{\linewidth}{
  {\parbox{\linewidth}{
  %%{\parbox{\linewidth}{
    %{\bf \textcolor{hdimdimgray}{
    {\bf {
            \sffamily
	    %\hfill Example \theHexpoctr \hfill}} }}}
	    Example \theHexpoctr \hfill}} }}
	    \\ 
	    %\textcolor{hdimdimgray}{\sffamily\noindent#1 $\square$} % \\ \centering $\square$
	    {\sffamily\noindent#1 $\square$} % \\ \centering $\square$
}{
  \medskip
}

% *----------------------------*
% | HOVEY'S Hquote ENVIRONMENT |
% *----------------------------*--------------------------------
\newenvironment{Hquote}{% Hquote open
  \def\FrameCommand{% FrameCommand open
    \hspace{10pt}%
    {\color{hdimdimgray}\vrule width 1pt}% left-side vertical line
    {\color{hlightgray}\vrule width 4pt}% effectively a left-side margin
    \colorbox{hlightgray}%
  }% FrameCommand close
  \MakeFramed{\advance\hsize-\width\FrameRestore}%
  \setlength{\leftskip}{10pt}% indent left side
  \setlength{\rightskip}{14pt}% indent left side
  %\noindent\hspace{-4pt}% no first paragraph indent
  %\noindent% no first paragraph indent
  %\begin{adjustwidth}{}{7pt}%
  \vspace{2pt}\vspace{2pt}%
}% Hquote intermediate close
{% Hquote intermediate open
  \vspace{5pt}% bottom margin 
  %\end{adjustwidth}
  \endMakeFramed%
  % \setlength{\leftskip}{0pt}% restore back to original, not needed b/c inside environment
}% Hquote close


% *--------------------------*
% | HOVEY'S Hbox ENVIRONMENT |
% *--------------------------*------------------------------------
 
\newenvironment{Hbox}[1]{
    \begin{center}
    \fbox{\parbox[b]{\linewidth}{#1}}
    \end{center}
    }{
    }

\newenvironment{Hboxvar}[2]{
    \begin{center}
    \parbox{#1}{
    \fbox{\parbox[b]{#1}{#2}
         } % close fbox
                       } % close parbox
    \end{center}
                           } % close newenvironment
%
% #1 is the width of the box
% #2 is the contents that goes in the Hbox


\newenvironment{Hboxnew}[2]{
  \begin{center}
    \fbox{
      \begin{minipage}{#1} #2
      \end{minipage}
         }                         % close fbox
  \end{center}  
}                                  % close newenvironment



% *-----------------------------*
% | HOVEY'S Hremark ENVIRONMENT |
% *-----------------------------*---------------------------------

\newcounter{Hremarkctr}
\renewcommand{\theHremarkctr}{\arabic{Hremarkctr}.}
\newenvironment{Hremark}
 {
  \stepcounter{Hremarkctr}%
  \medskip
  \noindent {\bf Remark \thesection.\theHremarkctr}
 }{
  \medskip
 }

% *---------------------------------*
% | HOVEY'S Hassumption ENVIRONMENT |
% *---------------------------------*-----------------------------
 
\newcounter{Hassumptionctr}
\renewcommand{\theHassumptionctr}{\arabic{Hassumptionctr}.}
\newenvironment{Hassumption}
 {
  \stepcounter{Hassumptionctr}%
  \medskip
  \noindent {\bf Assumption \thesection.\theHassumptionctr}
 }{
  \medskip
 }
 
 

% *---------------------------------*
% | HOVEY'S Hdefinition ENVIRONMENT |
% *---------------------------------*-----------------------------
 
\newcounter{Hdefinitionctr}
\renewcommand{\theHdefinitionctr}{\arabic{Hdefinitionctr}.}
\newenvironment{Hdefinition}
 {
  \stepcounter{Hdefinitionctr}%
  \medskip
  \noindent {\bf Definition \thesection.\theHdefinitionctr}
 }{
  \medskip
 }

% *---------------------------------*
% | HOVEY'S Hdefn ENVIRONMENT |
% *---------------------------------*-----------------------------
 
\newcounter{Hdefnctr}
\renewcommand{\theHdefnctr}{\arabic{Hdefnctr}.}
\newenvironment{Hdefn}[2]
 {
  \stepcounter{Hdefnctr}%
  \medskip
  \noindent {\bf Definition \thesection.\theHdefnctr \vspace{0.0in} #1} \\
  #2
 }{
  \medskip
 }


% *---------------------------------*
% | HOVEY'S Hquestion ENVIRONMENT |
% *---------------------------------*-----------------------------
 
\newcounter{Hquestionctr}
\renewcommand{\theHquestionctr}{\arabic{Hquestionctr}.}
\newenvironment{Hquestion}
 {
  \stepcounter{Hquestionctr}%
  \medskip
  \noindent {\bf Question \thesection.\theHquestionctr}
 }{
  \medskip
 }
